\documentclass[]{final_report}
\usepackage{graphicx}
\usepackage{hyperref}


%%%%%%%%%%%%%%%%%%%%%%
%%% Input project details
\def\studentname{Dr. G.C.M. Silvestre and Dr. M. \'O Cinn\'eide}
\def\projecttitle{Guidelines and Regulations}
\def\supervisorname{Your Supervisor Name}
\def\moderatorname{Your Moderator Name}


\begin{document}

\maketitle
\tableofcontents\pdfbookmark[0]{Table of Contents}{toc}\newpage

%%%%%%%%%%%%%%%%%%%%%%
%%% Your Abstract here

\begin{abstract}

\textbf{\textsl{This document is a layout and formatting template for your ASE project. It's there to help you, but change it as much as your want!}}

What is an abstract? The abstract should provide a short overview of your project that enables a reader to decide if your report is of interest to them or not. It should be concise, to-the-point and interesting. Avoid making it read like a verbose table of contents. Avoid references, jargon or acronyms, as the reader may not be familiar with them. An abstract usually contains a brief description of:

\begin{itemize}
\item The project and its context;
\item How the project work was carried out;
\item The major findings or results.
\end{itemize}

One paragraph is plenty. The main thing to remember is the principle that the abstract must be short, and a person reading it should be able to determine if they want to read more. For example, if your project involves building a compiler for Java, and a major section of your work is focussed on developing an efficient parser (rather than say code-generation), make this clear in the abstract. Then a reader who is interested in efficient parsing techniques knows that your report may be of interest to them.

\end{abstract}
\newpage


%%%%%%%%%%%%%%%%%%%%%%
%%% Acknowledgments

\chapter*{Acknowledgments}

In your Acknowledgments section, give credit to all the people who helped you in your project.

%%%%%%%%%%%%%%%%%%%%%%
%%% Introduction

\chapter{Introduction}


The project report is a very important part of your ASE project and its preparation and presentation should be of extremely high quality. Remember that a large portion of the marks for your project are awarded for this report. 


\chapter{\label{chapter2} Submission Policy}

\section{Submission of Project Code}

Lorem ipsum dolor sit amet, consectetur adipisicing elit, sed do eiusmod tempor incididunt ut labore et dolore magna aliqua. Ut enim ad minim veniam, quis nostrud exercitation ullamco laboris nisi ut aliquip ex ea commodo consequat. Duis aute irure dolor in reprehenderit in voluptate velit esse cillum dolore eu fugiat nulla pariatur. Excepteur sint occaecat cupidatat non proident, sunt in culpa qui officia deserunt mollit anim id est laborum.

\section{Submission of Project Report}

Lorem ipsum dolor sit amet, consectetur adipisicing elit, sed do eiusmod tempor incididunt ut labore et dolore magna aliqua. Ut enim ad minim veniam, quis nostrud exercitation ullamco laboris nisi ut aliquip ex ea commodo consequat. Duis aute irure dolor in reprehenderit in voluptate velit esse cillum dolore eu fugiat nulla pariatur. Excepteur sint occaecat cupidatat non proident, sunt in culpa qui officia deserunt mollit anim id est laborum.

\chapter{Preparing your Project Report}

{\bf There is advice on writing your report elsewhere on the ASE pages.} This report is purely to demonstrate the Latex template.

\section{Create a Report Structure}

Lorem ipsum dolor sit amet, consectetur adipisicing elit, sed do eiusmod tempor incididunt ut labore et dolore magna aliqua. Ut enim ad minim veniam, quis nostrud exercitation ullamco laboris nisi ut aliquip ex ea commodo consequat. Duis aute irure dolor in reprehenderit in voluptate velit esse cillum dolore eu fugiat nulla pariatur. Excepteur sint occaecat cupidatat non proident, sunt in culpa qui officia deserunt mollit anim id est laborum.

\begin{itemize}
\item first item;
\item second item;
\item third item.
\end{itemize}

Lorem ipsum dolor sit amet, consectetur adipisicing elit, sed do eiusmod tempor incididunt ut labore et dolore magna aliqua. Ut enim ad minim veniam, quis nostrud exercitation ullamco laboris nisi ut aliquip ex ea commodo consequat. Duis aute irure dolor in reprehenderit in voluptate velit esse cillum dolore eu fugiat nulla pariatur. Excepteur sint occaecat cupidatat non proident, sunt in culpa qui officia deserunt mollit anim id est laborum.

\section{Typical Structure of your Report}

Lorem ipsum dolor sit amet, consectetur adipisicing elit, sed do eiusmod tempor incididunt ut labore et dolore magna aliqua. Ut enim ad minim veniam, quis nostrud exercitation ullamco laboris nisi ut aliquip ex ea commodo consequat. Duis aute irure dolor in reprehenderit in voluptate velit esse cillum dolore eu fugiat nulla pariatur. Excepteur sint occaecat cupidatat non proident, sunt in culpa qui officia deserunt mollit anim id est laborum.

\subsection{Title page, Table of Contents, and Acknowledgements}

The title page should state at least the project title, the name of your supervisor(s), and your name of course. A Table of Contents is essential, but should be produced by the wordprocessing package you are using. In your Acknowledgments section, give credit to all the people who helped you in your project.

The Introduction chapter introduces the project and describes the general subject area of the project. Topics it may contain include:

\begin{itemize}
\item A discussion of the original aims of the project, and the modified aims if appropriate;
\item The scope of the project and a general justification for the work undertaken, perhaps providing a brief background description;
\item A description of the structure of the report, i.e., a road map for the reader.
\end{itemize}

\subsection{The Abstract}

The abstract should provide a short overview of your project that enables a reader to decide if your report is of interest to them or not. It should be concise, to-the-point and interesting. Avoid making it read like a verbose table of contents! Avoid references, jargon or acronyms, as the reader may not be familiar with them. An abstract usually contains a brief description of:

\begin{itemize}
\item The project and its context;
\item How the project work was carried out;
\item The major findings or results.
\end{itemize}

One paragraph is plenty! The main thing to remember is the principle that the abstract must be short, and a person reading it should be able to determine if they want to read more. For example, if your project involves building a compiler for Java, and a major section of your work is focussed on developing an efficient parser (rather than say code-generation), make this clear in the abstract. Then a reader who is interested in efficient parsing techniques knows that your report may be of interest to them.

\subsection{Chapter 1: Introduction}

This chapter introduces the project and describes the general subject area of the project. Topics it may contain include:

\begin{itemize}
\item A discussion of the original aims of the project, and the modified aims if appropriate;
\item The scope of the project and a general justification for the work undertaken, perhaps providing a brief background description;
\item A description of the structure of the report, i.e., a road map for the reader.
\end{itemize}

\subsection{Chapter 2: Background Research}

Lorem ipsum dolor sit amet, consectetur adipisicing elit, sed do eiusmod tempor incididunt ut labore et dolore magna aliqua. Ut enim ad minim veniam, quis nostrud exercitation ullamco laboris nisi ut aliquip ex ea commodo consequat. Duis aute irure dolor in reprehenderit in voluptate velit esse cillum dolore eu fugiat nulla pariatur. Excepteur sint occaecat cupidatat non proident, sunt in culpa qui officia deserunt mollit anim id est laborum.

\subsection{Chapters 3 and 4: The Core Chapters}

Lorem ipsum dolor sit amet, consectetur adipisicing elit, sed do eiusmod tempor incididunt ut labore et dolore magna aliqua. Ut enim ad minim veniam, quis nostrud exercitation ullamco laboris nisi ut aliquip ex ea commodo consequat. Duis aute irure dolor in reprehenderit in voluptate velit esse cillum dolore eu fugiat nulla pariatur. Excepteur sint occaecat cupidatat non proident, sunt in culpa qui officia deserunt mollit anim id est laborum.

\subsection{Chapter 5: Detailed Design and Implementation}

Lorem ipsum dolor sit amet, consectetur adipisicing elit, sed do eiusmod tempor incididunt ut labore et dolore magna aliqua. Ut enim ad minim veniam, quis nostrud exercitation ullamco laboris nisi ut aliquip ex ea commodo consequat. Duis aute irure dolor in reprehenderit in voluptate velit esse cillum dolore eu fugiat nulla pariatur. Excepteur sint occaecat cupidatat non proident, sunt in culpa qui officia deserunt mollit anim id est laborum.

\subsection{Chapter 6: Testing/Evaluation}

Lorem ipsum dolor sit amet, consectetur adipisicing elit, sed do eiusmod tempor incididunt ut labore et dolore magna aliqua. Ut enim ad minim veniam, quis nostrud exercitation ullamco laboris nisi ut aliquip ex ea commodo consequat. Duis aute irure dolor in reprehenderit in voluptate velit esse cillum dolore eu fugiat nulla pariatur. Excepteur sint occaecat cupidatat non proident, sunt in culpa qui officia deserunt mollit anim id est laborum.

 Consult~\cite{DAWSON:2000} for some excellent advice on how to present the results of your experiments.

\subsection{Chapter 7: Conclusions and Future Work}
Lorem ipsum dolor sit amet, consectetur adipisicing elit, sed do eiusmod tempor incididunt ut labore et dolore magna aliqua. Ut enim ad minim veniam, quis nostrud exercitation ullamco laboris nisi ut aliquip ex ea commodo consequat. Duis aute irure dolor in reprehenderit in voluptate velit esse cillum dolore eu fugiat nulla pariatur. Excepteur sint occaecat cupidatat non proident, sunt in culpa qui officia deserunt mollit anim id est laborum.

\subsection{References}

Lorem ipsum dolor sit amet, consectetur adipisicing elit, sed do eiusmod tempor incididunt ut labore et dolore magna aliqua. Ut enim ad minim veniam, quis nostrud exercitation ullamco laboris nisi ut aliquip ex ea commodo consequat. Duis aute irure dolor in reprehenderit in voluptate velit esse cillum dolore eu fugiat nulla pariatur. Excepteur sint occaecat cupidatat non proident, sunt in culpa qui officia deserunt mollit anim id est laborum.

\subsection{Appendices}

Lorem ipsum dolor sit amet, consectetur adipisicing elit, sed do eiusmod tempor incididunt ut labore et dolore magna aliqua. Ut enim ad minim veniam, quis nostrud exercitation ullamco laboris nisi ut aliquip ex ea commodo consequat. Duis aute irure dolor in reprehenderit in voluptate velit esse cillum dolore eu fugiat nulla pariatur. Excepteur sint occaecat cupidatat non proident, sunt in culpa qui officia deserunt mollit anim id est laborum.


\section{Order of Writing}

Lorem ipsum dolor sit amet, consectetur adipisicing elit, sed do eiusmod tempor incididunt ut labore et dolore magna aliqua. Ut enim ad minim veniam, quis nostrud exercitation ullamco laboris nisi ut aliquip ex ea commodo consequat. Duis aute irure dolor in reprehenderit in voluptate velit esse cillum dolore eu fugiat nulla pariatur. Excepteur sint occaecat cupidatat non proident, sunt in culpa qui officia deserunt mollit anim id est laborum.

\section{Other Advice}

This section contains a number of guidelines that are worth bearing in mind when writing.


\subsection{Continuity}

Lorem ipsum dolor sit amet, consectetur adipisicing elit, sed do eiusmod tempor incididunt ut labore et dolore magna aliqua. Ut enim ad minim veniam, quis nostrud exercitation ullamco laboris nisi ut aliquip ex ea commodo consequat. Duis aute irure dolor in reprehenderit in voluptate velit esse cillum dolore eu fugiat nulla pariatur. Excepteur sint occaecat cupidatat non proident, sunt in culpa qui officia deserunt mollit anim id est laborum.

\subsection{Presentation Issues}

Focus on expressing your ideas clearly. Part of your report is of course its physical layout and use of diagrams. Try not to put too much time into this. If you use the provided \LaTeX\ template, it will help you focus on the content as opposed to the form. Simple diagrams are fine, and avoid the use of colour unless it really contributes something in particular. Do not bother with tricks like adjusting spacing or margins or fonts in order to make your report bigger or smaller.

Most ASE reports will contain a mixture of figures and charts along with the main body of text. The figure caption should appear directly after the figure as seen in Figure~\ref{fig:logo} whereas a table caption will appear directly above the table. Figures, charts and tables should always be centered horizontally. 

\begin{figure}[h]
\centering
\fboxsep 2mm
\framebox{
	\includegraphics[width=6cm]{figures/logo.pdf} 
	\includegraphics[width=3cm]{figures/logo.pdf} 
	\includegraphics[width=1.5cm]{figures/logo.pdf} 
	\includegraphics[width=0.75cm]{figures/logo.pdf} 
	\includegraphics[width=0.375cm]{figures/logo.pdf}
}
\caption{\label{fig:logo} Logo of the UCD Department of Computer Science displayed at various size.}
\end{figure} 

If you wish to print a short excerpt of your source code,  ensure that you are using a fixed-width sans-serif font such as the Courier font. Your code will be properly indented and will appear as follows:

\begin{verbatim}
static public void main(String[] args) {
  try  {
    UIManager.setLookAndFeel(UIManager.getSystemLookAndFeelClassName());
  }
  catch(Exception e) {
    e.printStackTrace();
  }
  new WelcomeApp();
} 
\end{verbatim}


%%%% ADD YOUR BIBLIOGRAPHY HERE
% OR use Bibtex if you prefer
\newpage
\begin{thebibliography}{99}
\bibitem{DAWSON:2000} Christian Dawson. \emph{The Essence of Computing Projects -- A Student's Guide}. 192 pages. ISBN: 013021972X. Pearson Education, 2000.
\end{thebibliography}
\label{endpage}

\end{document}

\end{article}
