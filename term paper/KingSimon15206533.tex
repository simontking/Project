\documentclass[a4paper,11pt,article,oneside]{memoir}
\usepackage{ecis2015}
\usepackage{subcaption}
\usepackage[utf8]{inputenc}

%%% Enter Document Info here: %%%%%%%%%%%%%%%%%%%%%%%%%%%%%%%%%%%%%%%%%%%%%%%%

\maintitle{improving recommendations with trust data} % ← Don't use UPPERCASE here, we do that automatically.
\shorttitle{Improving Recommendations with Trust Data} % ← This goes into the header.
\category{Case Study} % ← Choose one by deleting the others.

\authors{% Separate authors by a "\par" or blank line.
	King, Simon, University College Dublin, Ireland, simon.king@ucdconnect.ie}

\shortauthors{King} % ← This goes into the header. 

%\addbibresource{bibliography.bib}
\begin{document}	
	\tableofcontents
	\chapter{Introduction}
	Recommender systems are used across the web to provide personalised suggestions based on user interactions within systems. In this case study, we examine various existing techniques that are used to improve recommender systems based on past user actions and trust relationships. We attempt to implement our own recommendation model.
	
	
	\newpage
	\chapter{Recommender Systems}
	
	
	Online Social Networks provide a rapidly expanding source of information which can be used to turn data on a users interests and preferences into recommendations for possible items the user might be interested in in the future. There is no one best method for recommendation systems, rather, a multitude of solutions exist which can be best applied based context and application of the recommendation, as well as the information available. {cite}
	
	\section{Modelling}
	When assessing the architecture and design to see if a system can meet performance objectives, two models can be used: the software execution model and the system execution model. 
	\subsection{Software Execution Model} 
	
	\section{Performance Control Principles}
	
	\subsection{Performance Objectives}

	\chapter{Conclusions}
	\newpage
	% If you are using BibTeX/Biber:
	\nocite{*}
%	\printbibliography
	
\end{document}
